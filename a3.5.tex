\documentclass[11pt]{article}

\usepackage{ml5}

\newenvironment{exercise}{\item}{}
% \newcommand{\solution}[1]{\textbf{Solution.} #1}
\newcommand{\solution}[1]{}


\newcommand{\w}{\omega}
\newcommand{\Int}{\mathbb{Z}}
\newcommand{\Compl}{\mathbb{C}}
% \newcommand{\abs}[1]{\lvert#1\rvert}
\newcommand{\norm}[1]{\lVert#1\rVert}
\newcommand{\bigpa}[1]{\bigl(#1\bigr)}
\newcommand{\Bigpa}[1]{\Bigl(#1\Bigr)}

\begin{document}

\begin{center}
	\Large Математические основы алгоритмов, весна 2024 \\
	\Large Задание \(3.5\). Графы, линейное программирование
\end{center}


\begin{enumerate}
\begin{exercise}
На лекциях рассматривался следующий алгоритм
для приближенного решения задачи
поиска наименьшего вершинного покрытия в графе. % $G = (V,E)$.
На каждом шаге алгоритм выбирает \emph{произвольное ребро} $(u,v)$,
добавляет вершины $u$ и $v$ к вершинному покрытию,
и удаляет из графа эти вершины вместе со всеми примыкающими к ним рёбрами.
Это повторяется вплоть до удаления всех рёбер.

Доказать, что такой алгоритм не будет $\alpha$-приближённым ни для какого $\alpha < 2$.
\end{exercise}

	
\begin{exercise}
Рассматривается следующий жадный алгоритм
для приближенного решения задачи
о поиске наименьшего вершинного покрытия в графе. % $G = (V,E)$.
На каждом шаге алгоритм выбирает \emph{вершину наибольшей степени},
добавляет её к вершинному покрытию,
и удаляет её вместе со всеми примыкающими к ней рёбрами.
Это повторяется вплоть до удаления всех вершин.

Доказать, что такой алгоритм не будет $2$-приближённым.
Будет ли алгоритм $\alpha$-приближённым для какой-то константы $\alpha$?
\end{exercise}


\begin{exercise}
Поток в сети называется \emph{ациклическим}, если подграф сети, 
состоящий из ребер с ненулевым значением потока --- ациклический.
Докажите, что для любого потока найдется ациклический поток с тем же значением, что и у исходного.
\end{exercise}


\begin{exercise}
Задана сеть 
(ориентированный граф с неотрицательными вещественными пропускными способностями на ребрах
и выделенными вершинами $s$, $t$)
и максимальный поток на ней.
Найдите эффективный алгоритм для проверки, является ли этот максимальный поток единственным.
\end{exercise}


\begin{exercise}
Предъявите такую несовместную задачу линейного программирования, что двойственная задача будет также несовместна.
\end{exercise}


\begin{exercise}
Пусть $G = (V, E)$~--- двудольный граф.
Рёберное покрытие $G$~--- это такое множество $X \subseteq E$,
что любая вершина $v \in V$ инцидентна хотя бы одному ребру из множества $X$.
\begin{enumerate}
\item
	Опишите задачу линейного программирования для нахождения минимального рёберного покрытия.
\item
	Докажите, что если граф регулярный, то в минимальном покрытии не более $\frac{|V|}{2}$ ребер.
\item
	Постройте двойственную задачу линейного программирования и изучите её комбинаторный смысл.
\end{enumerate}
\end{exercise}


\begin{exercise}
Пусть дан граф $G = (V, E)$, $|V| = n$, $|E| = m$, $x \in \mathbb{R}^m$. Рассмотрим следующую
задачу линейного программирования:
$\max \sum\limits_{e \in E} x_e$, при условии
\begin{align*}
	x_e &\geqslant 0
		&& (\forall e \in E) \\
	\sum_{e \in E_v} x_e &\leqslant 1
		&& (\forall v \in V)
\end{align*}
где $E_v$~--- множество ребер, инцидентных вершине $v$.
\begin{enumerate}
\item
	Какой <<физический>> смысл у данной задачи? А если вектор $x$ имеет целочисленные координаты?
\item
	Докажите, что если граф $G$ двудольный, то оптимум достигается в вершине с целочисленными
	координатами.
\end{enumerate}
\end{exercise}

\end{enumerate}

\end{document}
