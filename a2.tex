\documentclass[11pt]{article}

\usepackage{ml5}

\newenvironment{exercise}{\item}{}
% \newcommand{\solution}[1]{\textbf{Solution.} #1}
\newcommand{\solution}[1]{}


\newcommand{\w}{\omega}
\newcommand{\Int}{\mathbb{Z}}
\newcommand{\Compl}{\mathbb{C}}
% \newcommand{\abs}[1]{\lvert#1\rvert}
\newcommand{\norm}[1]{\lVert#1\rVert}
\newcommand{\bigpa}[1]{\bigl(#1\bigr)}
\newcommand{\Bigpa}[1]{\Bigl(#1\Bigr)}


\begin{document}

\begin{center}
	\Large Математические основы алгоритмов, весна 2024 \\
	\Large Задание 2. Сети сортировки, параллельные схемы
\end{center}

\begin{enumerate}

\begin{exercise}
Докажите корректность схем битонного слияния и битонной сортировки, представленных в лекциях.
\end{exercise}


\begin{exercise}
Пусть имеется сеть сортировки. Останется ли она сетью сортировки, если

\begin{enumerate}
\item перевернуть ее диаграмму относительно вертикальной оси?
(при этом направление сортировки пар индивидуальными элементами сравнения не переворачивается)?

\item перевернуть ее диаграмму относительно горизонтальной оси?

\item добавить к сети произвольный элемент сравнения?
\end{enumerate}
\end{exercise}


\begin{exercise}

\emph{Сетью транспозиции} называется сеть сравнения, 
в диаграмме которой все попарные сравнения производятся между соседними проводами.

\begin{enumerate}
\item Докажите, что если сеть транспозиции на $n$ входах является сетью сортировки,
то в ней не менее $\binom{n}{2}$ элементов сравнения.

\item Докажите, что если сеть транспозиции на $n$ входах сортирует последовательность
$(n, n-1, \ldots 2, 1)$ (то есть последовательность, отсортированную по убыванию),
то она является сетью сортировки.
\end{enumerate}
\end{exercise}


\begin{exercise}
Ответьте на вопросы упражнения 2 при условии, что упоминаемая в них сеть является сетью транспозиции.
\end{exercise}


\begin{exercise}
\emph{Правильная скобочная последовательность} определяется рекурсивно:
это либо пустая последовательность, либо последовательность `($S$)', либо последовательность `$ST$',
где $S$, $T$ --- правильные скобочные последовательности.
Например, скобочные последовательности `(((())))', `()(())()((()))', `' --- правильные, 
а последовательности `)(', `(()))' --- неправильные. 
Опишите эффективную параллельную схему распознавания правильных скобочных последовательностей.
\end{exercise}


\begin{exercise}
В коде на языке Python, комментарий состоит из символа `\texttt{\#}' 
и всех следующих за ним символов, вплоть до конца строки.
(Комментарий может, в числе прочих, содержать произвольное количество символов `\texttt{\#}').
Задан массив из $n$ символов, содержащий код на Python;
в нем концы строк обозначены специальным символом `$\hookleftarrow$', 
ограничения на длину строки отсутствуют. 
Опишите эффективную параллельную схему, заменяющий все символы комментариев на пробелы.
\end{exercise}


\begin{exercise}
Система линейных уравнений $Ax = b$ задана $n \times n$ матрицей $A$ и $n$-вектором $b$.
Матрица $A$ является \emph{трехдиагональной}, то есть $a_{ij}=0$ для всех $i$, $j$, где $\abs{i-j}>1$.
Опишите эффективную параллельную схему для решения такой системы
\begin{enumerate}
\item предполагая, что $a_{ij} \neq 0$ при $\abs{i-j} \leq 1$
\item* в общем случае
\end{enumerate}
Предполагается, что решение существует и единственно, и что все вычисления точные.
\end{exercise}


\begin{exercise}
Опишите эффективный параллельный алгоритм для быстрого преобразования Фурье в произвольном кольце $R$
(удовлетворяющем условиям существования DFT):
%
\begin{enumerate}
%
\item как схему с элементарными операциями сложения и умножения в $R$;
%
\item в модели BSP с оптимальной трудоемкостью и коммуникацией, 
и минимально возможным количеством синхронизаций;
при этом можно предположить, что степень преобразования $n$ достаточно велика 
по сравнению с количеством процессоров $p$.
%
\end{enumerate}
\end{exercise}


\begin{exercise}*
Опишите эффективный параллельный алгоритм для сортировки массива из $n$ элементов 
произвольного линейно упорядоченного множества
с элементарной операцией сравнения двух элементов,
в модели BSP с оптимальной трудоемкостью и коммуникацией, 
и минимально возможным количеством синхронизаций;
при этом можно предположить, что размер массива $n$ достаточно велик 
по сравнению с количеством процессоров $p$.
\end{exercise}

\end{enumerate}

\end{document}
