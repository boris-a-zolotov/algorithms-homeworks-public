\documentclass[11pt]{article}

\usepackage{ml5}

\newenvironment{exercise}{\item}{}
% \newcommand{\solution}[1]{\textbf{Solution.} #1}
\newcommand{\solution}[1]{}


\newcommand{\w}{\omega}
\newcommand{\Int}{\mathbb{Z}}
\newcommand{\Compl}{\mathbb{C}}
% \newcommand{\abs}[1]{\lvert#1\rvert}
\newcommand{\norm}[1]{\lVert#1\rVert}
\newcommand{\bigpa}[1]{\bigl(#1\bigr)}
\newcommand{\Bigpa}[1]{\Bigl(#1\Bigr)}

\begin{document}

\begin{center}
	\Large Математические основы алгоритмов, весна 2024 \\
	\Large Задание \(3.9\). Задача коммивояжёра. Проверка умножения матриц
\end{center}


\begin{enumerate}

\begin{exercise}
Рассматривается задача коммивояжёра, в которой все вершины графа размещены на плоскости,
и расстояния между ними равны длинам отрезков.
Ни один из отрезков, соединяющих пары вершин, не проходит ни через какую третью вершину.
Показать, что оптимальный цикл не содержит самопересечений.

\end{exercise}

\begin{exercise}
Рассматривается следующая эвристика для решения задачи коммивояжера
в предположении, что расстояния на рёбрах удовлетворяют неравенству треугольника.
Алгоритм начинает с цикла длины 1 --- произвольной вершины.
На каждом шаге алгоритм находит вершину $u$, не принадлежащую текущему циклу,
расстояние от которой до ближайшей вершины $v$ цикла минимально,
и вставляет $u$ в цикл сразу после $v$.

Доказать, что такой алгоритм будет 2-приближенным,
и не будет $\alpha$-приближенным ни для какого $\alpha < 2$.
\end{exercise}

\begin{exercise}
Предположим, что алгоритм двойного дерева имеет доступ к оракулу,
который по полученному алгоритмом эйлерову графу 
выбирает его оптимальный обход и оптимальную последовательность сокращений этого обхода.
Докажите, что даже в этом случае алгоритм двойного дерева 
для задачи коммивояжера с неравенством треугольника 
не будет $\alpha$-приближённым ни для какого $\alpha<5/3$.
\end{exercise}

\begin{exercise}
Предположим, что алгоритм Кристофидеса--Сердюкова имеет доступ к оракулу,
аналогичному описанному в предыдущей задаче.
Докажите, что даже в этом случае алгоритм Кристофидеса--Сердюкова 
для задачи коммивояжера с неравенством треугольника
не будет $\alpha$-приближённым ни для какого $\alpha<3/2$.
\end{exercise}

\begin{exercise}
Даны три числовых матрицы:
$A=\left(\begin{matrix} 2 & 6 \\ 0 & 4 \end{matrix}\right)$,
$B=\left(\begin{matrix} 2 & 0 \\ 2 & 1 \end{matrix}\right)$ 
и
$C=\left(\begin{matrix} 15 & 7 \\ 8 & 4 \end{matrix}\right)$ 

С какой вероятностью алгоритм Фрейвальдса сообщит о несовпадении $A \times B$ и $C$?
\end{exercise}

\begin{exercise}
Даны три матрицы над кольцом: $A, B, C \in \mathbb{R}^{n \times n}$,
причём известно, что $C$ отличается от $A \times B$ не более чем в одной клетке.
Построить детерминированный алгоритм, исправляющий возможную ошибку,
то есть вычисляющий $A \times B$ за время $O(n^2)$.
\end{exercise}



\end{enumerate}

\end{document}
