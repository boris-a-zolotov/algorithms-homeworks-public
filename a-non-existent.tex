\documentclass[11pt]{article}

\usepackage{ml5}

\newenvironment{exercise}{\item}{}
% \newcommand{\solution}[1]{\textbf{Solution.} #1}
\newcommand{\solution}[1]{}


\newcommand{\w}{\omega}
\newcommand{\Int}{\mathbb{Z}}
\newcommand{\Compl}{\mathbb{C}}
% \newcommand{\abs}[1]{\lvert#1\rvert}
\newcommand{\norm}[1]{\lVert#1\rVert}
\newcommand{\bigpa}[1]{\bigl(#1\bigr)}
\newcommand{\Bigpa}[1]{\Bigl(#1\Bigr)}

\begin{document}

\begin{center}
	\Large Математические основы алгоритмов, весна 2024 \\
	\Large Задание 3.7. Приближённые алгоритмы. Задача коммивояжёра. Задача о наибольшем паросочетании
\end{center}

\begin{description}

% выравнивание пробелами \ и \, нужно исправить

\begin{exercise}[\ \ \ -2]
Доказать $2$-приближённость алгоритма двойного дерева.

\emph{Алгоритм двойного дерева.} Приближённый алгоритм для задачи коммивояжёра. Рёбра минимального остовного дерева удваиваются и получается Эйлеров цикл. При обходе цикла пропускаем повторяющиеся вершины.
\end{exercise}

\begin{exercise}[-3/2]
Доказать $\frac{3}{2}$-приближённость алгоритма Кристофидеса-Сердюкова.

\emph{Алгоритм Кристофидеса-Сердюкова.} Приближённый алгоритм для задачи коммивояжёра. Создаём минимальное остовное дерево $T$. Находим в нём нечётные вершины и индуцируем граф $O$ на них. Находим совершенное паросочетание $M$ минимального веса для $O$. Дублируем рёбра из $M$ в $T$, получая мультиграф. Теперь в нём все вершины имеют чётную степень, а значит можем найти Эйлеров цикл. При обходе цикла пропускаем повторяющиеся вершины.
\end{exercise}

\begin{exercise}[\ \, {-}{-}{-}]
Доказать \emph{лемму Бержа}: паросочетание $M$ в графе $G$ является наибольшим тогда и только тогда, когда не существует дополняющего пути (пути, который начинается и завершается на свободных, то есть не принадлежащих паросочетаниям, вершинах и поочерёдно проходит по рёбрам, принадлежащим и не принадлежащим паросочетанию) в $M$.
\end{exercise}

\begin{exercise}[\,bip.]
Оптимизировать blossom algorithm на двудольном графе.

%\emph{Blossom algorithm.}
\end{exercise}

\begin{exercise}[$\mathbf{EV^2}$]

Доказать, что blossom algorithm работает за $O(EV^2)$ времени.

\end{exercise}

\end{description}

\end{document}
