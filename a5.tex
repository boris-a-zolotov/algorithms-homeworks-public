\documentclass[a4paper,11pt]{article}
\usepackage{ml5}

\begin{document}

\begin{center}
	\Large Математические основы алгоритмов, весна 2024 \\
	\Large Задание 5. MST, EMST, вероятностные алгоритмы
\end{center}

\begin{enumerate}

	\item {\it (Из лекций)} Привести пример серии графов, размер которых стремится к бесконечности и на которых алгоритм Борувки требует времени порядка \(|E| \cdot \log |V|\).

	\item Пусть в одном графе есть два различных MST, \(T_1\) и \(T_2\). Доказать, что если взять списки весов рёбер этих деревьев и отсортировать каждый из них, то получатся одинаковые результаты.

	\item Задача {\it MAX-3SAT:} дана формула в 3-КНФ; все переменные (под отрицанием или без) внутри одного клоза различны. Требуется означить переменные таким образом, чтобы удовлетворить как можно больше клозов. Доказать, что {\it случайное} означивание переменных даёт результат не более чем в \(\frac87\) раз хуже оптимального.

	\item Дан граф \(G\) и число \(k\); надо проверить, есть ли в \(G\) простой путь длины \(k\). Сделать это за время \(c^k \cdot n^{O(1)}\) с вероятностью ошибки не больше \(\tfrac{1}{3}\). Подсказка: случайно раскрасим вершины в два цве\ldots

	\item Пусть в графе \(G\) ребро \(e\)~— самое тяжёлое ребро на некотором цикле. Доказать, что существует MST для \(G\), которое не содержит \(e\).

	\item Рассмотрим алгоритм {\it „разделяй и властвуй”} для задачи MST: разбиваем вершины графа на два множества примерно одинакового размера, ищем MST для каждой из двух половинок, находим самое короткое ребро между двумя половинками. Всегда ли такой алгоритм возвращает корректный ответ?

	\item Евклидово минимальное остовное дерево (EMST)~— минимальное остовное дерево для точек на плоскости, вес ребра между которыми~— длина отрезка на плоскости. Каждое ребро некоторого EMST подразбили вершиной ровно посередине. Обязательно ли получилось EMST для нового набора вершин? А если все рёбра подразбиты в одном отношении, но не посередине?

	\item Трингуляция Делоне (DT)~— триангуляция множества точек \(\left\{ s_1, \ldots, s_n \right\}\) такая, что для каждого треугольника \( s_i s_j s_k \) все остальные точки лежат вне его описанного круга. Доказать, что EMST~— подграф DT.

	\item Рассматривается полное троичное сбалансированное дерево высоты \(h\). В его листьях записана последовательность из \(3^h\) нулей и единиц, а каждый узел вычисляет majority элемент своих детей. Нужно вычислить значение в корне. Доказать, что детерминированный алгоритм должен знать все \(3^h\) значений. Построить вероятностный алгоритм, всегда возвращающий правильный ответ, с ожидаемым временем работы \(C^h\), \(C < 3\).

\end{enumerate}
	
\end{document}
