\documentclass[a4paper,11pt]{article}
\usepackage{ml5}

\begin{document}

\begin{center}
	\Large Математические основы алгоритмов, весна 2024 \\
	\Large Задание 6. Борувка за линию, онлайн
\end{center}

\begin{enumerate}

	\item Заставьте алгоритм Борувки работать на планарном графе за \(O \lr*{|E|}\): докажите очевидную оценку и добавьте необходимые детали, чтобы не рассматривать на каждом шаге рёбра, стянувшиеся в петли.

	\item Предложите полиномиальный вероятностный алгоритм для задачи поиска максимального паросочетания.
	% https://codeforces.com/blog/entry/92400

	\item Хэш Пирсона последовательности байтов \(s_0 \ldots s_{n-1}\) определяется следующим образом. Пусть \(\sigma \in S_{256}\), \(h_0 = 0\), \[h_{i+1} \coloneqq \sigma\lr*{h_i \oplus s_i}.\] Значение хэша — \(h_n\). Будет ли набор хэшей Пирсона с каким-либо разумным набором перестановок универсальным семейством хэш-функций?
	
	\item {\it Онлайн—список поиска.} Рассмотрим следующую модель: двусвязный список, элементы в котором ищутся проходом к ним от начала списка. Соседние элементы можно менять местами, для этого до них надо дойти от элемента, который только что искали.

	Известна последовательность запросов: элементов, которые нужно найти. Точнее, известна она только гипотетическому оффлайн-алгоритму, который, к тому же, знает, какие элементы куда двигать после очередного запроса, чтобы минимизировать общее число шагов поиска и перестановок элементов.

	Нам же элементы для поиска выдаются по очереди, онлайн. Предложите алгоритм, который ищет очередной элемент и производит некоторые перестановки в двусвязном списке так, чтобы общее количество операций было достаточно близко к оптимальному.

	\item {\it Онлайн—кэширование.} Доказать, что алгоритм, который выбрасывает из кэша элемент, появившийся там раньше остальных, делает запросы к элементам не из кэша не более чем в \(O\lr*{k}\) раз чаще, чем воображаемый оптимальный оффлайн-алгоритм.

\end{enumerate}

\end{document}
