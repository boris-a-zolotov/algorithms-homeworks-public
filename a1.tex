\documentclass[11pt]{article}

\usepackage{ml5}

\newenvironment{exercise}{\item}{}

\newcommand{\w}{\omega}
\newcommand{\Int}{\mathbb{Z}}
\newcommand{\Compl}{\mathbb{C}}
\newcommand{\norm}[1]{\lVert#1\rVert}
\newcommand{\bigpa}[1]{\bigl(#1\bigr)}
\newcommand{\Bigpa}[1]{\Bigl(#1\Bigr)}

\begin{document}

\begin{center}
	\Large Математические основы алгоритмов, весна 2024 \\
	\Large Задание 1. Фурье, свёртки, Карацуба
\end{center}


\begin{enumerate}

\begin{exercise}
По данным аргументам $a_0 < a_1 < \ldots < a_{n-1}$
за $O(n\, log^2\, n)$ вычислить коэффициенты полинома степени $n$ с нулями ровно в этих аргументах.
\end{exercise}

\begin{exercise}
Пусть известно преобразование Фурье \(n\)-мерного вектора \([a_0, \ldots, a_{n-1}]\).
Как за \(O(n)\) найти преобразование Фурье \([a_1, \ldots, a_n]\)?
\end{exercise}

\begin{exercise}
Докажите корректность метода быстрого вычисления циклической и косоциклической сверток,
представленного в лекциях:
%
\begin{gather*}
\textstyle a \circledast_+ b = \frac{1}{n} F_{n,\w^{-1}}\bigpa{(F_{n,\w}a)\circ (F_{n,\w}b)}\\
%
\textstyle a \circledast_- b = w_{n,\psi^{-1}} \circ 
\Bigpa{\frac{1}{n} F_{n,\w^{-1}}\bigpa{
  (F_{n,\w} (w_{n,\psi} \circ a)\circ (F_{n,\w} (w_{n,\psi} \circ b))}}
\end{gather*}
%
где \(a \circledast_{\pm} b\)~— вектор, коэффициент на \(i\)-той позиции которого равен
%
\begin{gather*}
  \sum_{0 \leq j \leq i} a_j b_{i-j} \pm 
  \sum_{i < j < n} a_j b_{n+i-j},\quad 0 \leq i < n\\
\text{$\psi$ --- первообразный корень из единицы степени $2n$},
\psi^2 = \w\\
w_{n,\psi} = (1,\psi,\ldots,\psi^{n-1})^T
\end{gather*}
\end{exercise}

\begin{exercise}
Схему алгоритма Карацубы для умножения полиномов можно представить в матричном виде.
Пусть 
{\small
\begin{gather*} 
a(x) = a'(x) + a''(x) x^m\\
b(x) = b'(x) + b''(x) x^m\\ 
c(x) = a(x) \cdot b(x) = c'(x) + c''(x) x^m + c'''(x) x^{2m}
\end{gather*}}
%
Тогда
{\small
\begin{gather*} 
\begin{pmatrix}
c'(x) \\ c''(x) \\ c'''(x)
\end{pmatrix}
= 
P \cdot
\left(\left(
Q \cdot
\begin{pmatrix}
a'(x) \\ a''(x)
\end{pmatrix}
\right)
\circ
\left(
Q \cdot
\begin{pmatrix}
b'(x) \\ b''(x)
\end{pmatrix}
\right)\right)
\end{gather*}}
где
{\small
\begin{gather*} 
P = 
\begin{pmatrix}
1 & 0 & 0 \\ -1 & 1 & -1 \\ 0 & 0 & 1
\end{pmatrix}
\qquad
Q =
\begin{pmatrix}
1 & 0 \\ 1 & 1 \\ 0 & 1
\end{pmatrix}
\end{gather*}}
%
\begin{enumerate}
%
\item Опишите альтернативную рекурсивную схему (а именно, матрицу $P$), 
имеющую ту же трудоемкость, что и алгоритм Карацубы, где 
{\small
\begin{gather*} 
Q =
\begin{pmatrix}
1 & 0 \\ 1 & 1 \\ 1 & -1
\end{pmatrix}
\end{gather*}}
%
\item Объясните вид матрицы $Q$ и соотношение между матрицами $P$ и $Q$ 
в исходной и альтернативной схемах алгоритма Карацубы.  % Вандермонд для infty
%
\item В алгоритме Карацубы каждый из перемножаемых полиномов из $n$ членов 
разбивается на два полинома из $n/2$ членов, 
и между полученными меньшими полиномами выполняется три умножения.
Опишите рекурсивную схему для умножения полиномов,
где каждый из перемножаемых полиномов из $n$ членов 
разбивается на три полинома из $n/3$ членов, 
и между полученными меньшими полиномами выполняется пять умножений.
Дает ли эта схема выигрыш в трудоемкости по сравнению с алгоритмом Карацубы?
%
\end{enumerate}

\end{exercise}

\begin{exercise}
В анализе схемы алгоритма Карацубы для умножения двоичных чисел, 
в отличие от умножения полиномов, имеется одна тонкость, опущенная в лекции.
Пусть 
{\small
\begin{gather*} 
a = a' + a'' \boxdot 2^m\qquad
b = b' + b'' \boxdot 2^m
\end{gather*}}%
%
где $a$, $b$ --- числа из $n$ бит, $a'$, $a''$, $b'$, $b''$ --- числа из $n/2$ бит.
Одно из произведений, вычисляемых в ходе алгоритма, равняется $(a' + a'') \boxdot (b' + b'')$.
В отличие от умножения полиномов, неверно утверждать, 
что в каждом из сомножителей этого произведения $n/2$ бит:
количество битов может быть $n/2+1$ за счет переноса при сложении.
Объясните, как скорректировать анализ трудоемкости алгоритма, чтобы учесть этот факт.
\end{exercise}

\begin{exercise}
Определим прямое и обратное \emph{нормированное дискретное преобразование Фурье (DFT)} вектора $a \in\Compl^n$,
как $\frac{1}{\sqrt{n}} F_{n,\w} \cdot a$ и $\frac{1}{\sqrt{n}} F_{n,\w^{-1}} \cdot a$, 
соответственно. 
%
\begin{enumerate}
%
\item Докажите, что прямое и обратное DFT действительно взаимно обратны
и в обычном (с множителями $1$, $\frac{1}{n}$), и в нормированном 
(с множителями $\frac{1}{\sqrt{n}}$, $\frac{1}{\sqrt{n}}$) определении.
%
\item Пусть $a,b \in\Compl^n$, вектор $b$ --- нормированное DFT вектора $a$.
Докажите \emph{теорему Парсеваля}: $\norm{a} = \norm{b}$, 
где $\norm{x} = \bigpa{\sum_{0 \leq i < n} \abs{x_i}^2}^{1/2}$.
%
\item При каких $k$, $k$-кратное применение нормированного прямого DFT
с фиксированными $n$, $\w$ равносильно нормированному обратному DFT? 
Тождественному преобразованию?
%
\end{enumerate}
\end{exercise}

\begin{exercise}
    Обобщите алгоритм поиска подстроки с помощью FFT на случай образца, содержащего «джокеры» \verb!?!, соответствующие любому символу текста. А также на случай, когда текст также может содержать джокеры.
\end{exercise}

\begin{exercise}
    Даны два \(n\)-мерных вектора \(a\), \(b\). За $O(n\, log\, n)$ найти циклический сдвиг вектора \(b\), скалярное произведение которого с вектором \(a\) будет наибольшим.
\end{exercise}

\end{enumerate}

\end{document}
